\documentclass[]{article}
\usepackage{lmodern}
\usepackage{amssymb,amsmath}
\usepackage{ifxetex,ifluatex}
\usepackage{fixltx2e} % provides \textsubscript
\ifnum 0\ifxetex 1\fi\ifluatex 1\fi=0 % if pdftex
  \usepackage[T1]{fontenc}
  \usepackage[utf8]{inputenc}
\else % if luatex or xelatex
  \ifxetex
    \usepackage{mathspec}
  \else
    \usepackage{fontspec}
  \fi
  \defaultfontfeatures{Ligatures=TeX,Scale=MatchLowercase}
\fi
% use upquote if available, for straight quotes in verbatim environments
\IfFileExists{upquote.sty}{\usepackage{upquote}}{}
% use microtype if available
\IfFileExists{microtype.sty}{%
\usepackage{microtype}
\UseMicrotypeSet[protrusion]{basicmath} % disable protrusion for tt fonts
}{}
\usepackage[margin=1in]{geometry}
\usepackage{hyperref}
\hypersetup{unicode=true,
            pdftitle={HW3},
            pdfauthor={Erika Bueno and Zoe Portlas},
            pdfborder={0 0 0},
            breaklinks=true}
\urlstyle{same}  % don't use monospace font for urls
\usepackage{color}
\usepackage{fancyvrb}
\newcommand{\VerbBar}{|}
\newcommand{\VERB}{\Verb[commandchars=\\\{\}]}
\DefineVerbatimEnvironment{Highlighting}{Verbatim}{commandchars=\\\{\}}
% Add ',fontsize=\small' for more characters per line
\usepackage{framed}
\definecolor{shadecolor}{RGB}{248,248,248}
\newenvironment{Shaded}{\begin{snugshade}}{\end{snugshade}}
\newcommand{\KeywordTok}[1]{\textcolor[rgb]{0.13,0.29,0.53}{\textbf{#1}}}
\newcommand{\DataTypeTok}[1]{\textcolor[rgb]{0.13,0.29,0.53}{#1}}
\newcommand{\DecValTok}[1]{\textcolor[rgb]{0.00,0.00,0.81}{#1}}
\newcommand{\BaseNTok}[1]{\textcolor[rgb]{0.00,0.00,0.81}{#1}}
\newcommand{\FloatTok}[1]{\textcolor[rgb]{0.00,0.00,0.81}{#1}}
\newcommand{\ConstantTok}[1]{\textcolor[rgb]{0.00,0.00,0.00}{#1}}
\newcommand{\CharTok}[1]{\textcolor[rgb]{0.31,0.60,0.02}{#1}}
\newcommand{\SpecialCharTok}[1]{\textcolor[rgb]{0.00,0.00,0.00}{#1}}
\newcommand{\StringTok}[1]{\textcolor[rgb]{0.31,0.60,0.02}{#1}}
\newcommand{\VerbatimStringTok}[1]{\textcolor[rgb]{0.31,0.60,0.02}{#1}}
\newcommand{\SpecialStringTok}[1]{\textcolor[rgb]{0.31,0.60,0.02}{#1}}
\newcommand{\ImportTok}[1]{#1}
\newcommand{\CommentTok}[1]{\textcolor[rgb]{0.56,0.35,0.01}{\textit{#1}}}
\newcommand{\DocumentationTok}[1]{\textcolor[rgb]{0.56,0.35,0.01}{\textbf{\textit{#1}}}}
\newcommand{\AnnotationTok}[1]{\textcolor[rgb]{0.56,0.35,0.01}{\textbf{\textit{#1}}}}
\newcommand{\CommentVarTok}[1]{\textcolor[rgb]{0.56,0.35,0.01}{\textbf{\textit{#1}}}}
\newcommand{\OtherTok}[1]{\textcolor[rgb]{0.56,0.35,0.01}{#1}}
\newcommand{\FunctionTok}[1]{\textcolor[rgb]{0.00,0.00,0.00}{#1}}
\newcommand{\VariableTok}[1]{\textcolor[rgb]{0.00,0.00,0.00}{#1}}
\newcommand{\ControlFlowTok}[1]{\textcolor[rgb]{0.13,0.29,0.53}{\textbf{#1}}}
\newcommand{\OperatorTok}[1]{\textcolor[rgb]{0.81,0.36,0.00}{\textbf{#1}}}
\newcommand{\BuiltInTok}[1]{#1}
\newcommand{\ExtensionTok}[1]{#1}
\newcommand{\PreprocessorTok}[1]{\textcolor[rgb]{0.56,0.35,0.01}{\textit{#1}}}
\newcommand{\AttributeTok}[1]{\textcolor[rgb]{0.77,0.63,0.00}{#1}}
\newcommand{\RegionMarkerTok}[1]{#1}
\newcommand{\InformationTok}[1]{\textcolor[rgb]{0.56,0.35,0.01}{\textbf{\textit{#1}}}}
\newcommand{\WarningTok}[1]{\textcolor[rgb]{0.56,0.35,0.01}{\textbf{\textit{#1}}}}
\newcommand{\AlertTok}[1]{\textcolor[rgb]{0.94,0.16,0.16}{#1}}
\newcommand{\ErrorTok}[1]{\textcolor[rgb]{0.64,0.00,0.00}{\textbf{#1}}}
\newcommand{\NormalTok}[1]{#1}
\usepackage{graphicx,grffile}
\makeatletter
\def\maxwidth{\ifdim\Gin@nat@width>\linewidth\linewidth\else\Gin@nat@width\fi}
\def\maxheight{\ifdim\Gin@nat@height>\textheight\textheight\else\Gin@nat@height\fi}
\makeatother
% Scale images if necessary, so that they will not overflow the page
% margins by default, and it is still possible to overwrite the defaults
% using explicit options in \includegraphics[width, height, ...]{}
\setkeys{Gin}{width=\maxwidth,height=\maxheight,keepaspectratio}
\IfFileExists{parskip.sty}{%
\usepackage{parskip}
}{% else
\setlength{\parindent}{0pt}
\setlength{\parskip}{6pt plus 2pt minus 1pt}
}
\setlength{\emergencystretch}{3em}  % prevent overfull lines
\providecommand{\tightlist}{%
  \setlength{\itemsep}{0pt}\setlength{\parskip}{0pt}}
\setcounter{secnumdepth}{0}
% Redefines (sub)paragraphs to behave more like sections
\ifx\paragraph\undefined\else
\let\oldparagraph\paragraph
\renewcommand{\paragraph}[1]{\oldparagraph{#1}\mbox{}}
\fi
\ifx\subparagraph\undefined\else
\let\oldsubparagraph\subparagraph
\renewcommand{\subparagraph}[1]{\oldsubparagraph{#1}\mbox{}}
\fi

%%% Use protect on footnotes to avoid problems with footnotes in titles
\let\rmarkdownfootnote\footnote%
\def\footnote{\protect\rmarkdownfootnote}

%%% Change title format to be more compact
\usepackage{titling}

% Create subtitle command for use in maketitle
\newcommand{\subtitle}[1]{
  \posttitle{
    \begin{center}\large#1\end{center}
    }
}

\setlength{\droptitle}{-2em}

  \title{HW3}
    \pretitle{\vspace{\droptitle}\centering\huge}
  \posttitle{\par}
    \author{Erika Bueno and Zoe Portlas}
    \preauthor{\centering\large\emph}
  \postauthor{\par}
      \predate{\centering\large\emph}
  \postdate{\par}
    \date{January 28, 2019}


\begin{document}
\maketitle

In teams of two, address the following questions:

What species (and location if available) did you choose? Why did you
choose the species? We chose the queen alexendra sulphur butterfly
(Colias alexandra) because we wanted to understand its population
status.

What question do you want to answer about this population
(e.g.~population status, best management strategies)?\\
The question want to answer is: What life stage (egg, 1st instar, 2nd
instar, 3rd instar, diapause (pupae), post diapause, adult) is most
important for maintaining the growth of the population?

Calculate eigenvalue, stable age distribution, elasticity, and
sensitivity. What does this tell you about the population?

Using the calculations in part (c), or additional calculations, address
the question you proposed in part (b).

Submit your writeup as an R markdown file on Github. This can be a
private or public repository. I expect to see commits to the repository
from each partner.

\begin{Shaded}
\begin{Highlighting}[]
\KeywordTok{load}\NormalTok{(}\StringTok{"C:/Users/ebuen/Desktop/QuantReasoning/COMADRE_v.2.0.1.RData"}\NormalTok{)}
\KeywordTok{grep}\NormalTok{(comadre}\OperatorTok{$}\NormalTok{metadata}\OperatorTok{$}\NormalTok{Order, }\DataTypeTok{pattern=} \StringTok{'Lepidoptera'}\NormalTok{) }\CommentTok{#search by order}
\end{Highlighting}
\end{Shaded}

\begin{verbatim}
##  [1] 814 815 816 817 818 819 820 845 846 847 848 849 850 851 852
\end{verbatim}

\begin{Shaded}
\begin{Highlighting}[]
\NormalTok{comadre}\OperatorTok{$}\NormalTok{mat[}\DecValTok{845}\NormalTok{][[}\DecValTok{1}\NormalTok{]]}\OperatorTok{$}\NormalTok{matA }\CommentTok{#Alexandras sulphur butterfly with 7 life stages}
\end{Highlighting}
\end{Shaded}

\begin{verbatim}
##          A1    A2     A3    A4    A5    A6     A7
## [1,] 0.0000 0.000 0.0000 0.000 0.000 0.000 705.75
## [2,] 0.6218 0.000 0.0000 0.000 0.000 0.000   0.00
## [3,] 0.0000 0.598 0.0000 0.000 0.000 0.000   0.00
## [4,] 0.0000 0.000 0.5666 0.000 0.000 0.000   0.00
## [5,] 0.0000 0.000 0.0000 0.488 0.000 0.000   0.00
## [6,] 0.0000 0.000 0.0000 0.000 0.217 0.000   0.00
## [7,] 0.0000 0.000 0.0000 0.000 0.000 0.725   0.00
\end{verbatim}

\begin{Shaded}
\begin{Highlighting}[]
\NormalTok{aMatrix<-comadre}\OperatorTok{$}\NormalTok{mat[}\DecValTok{845}\NormalTok{][[}\DecValTok{1}\NormalTok{]]}\OperatorTok{$}\NormalTok{matA}
\KeywordTok{print}\NormalTok{(aMatrix)}
\end{Highlighting}
\end{Shaded}

\begin{verbatim}
##          A1    A2     A3    A4    A5    A6     A7
## [1,] 0.0000 0.000 0.0000 0.000 0.000 0.000 705.75
## [2,] 0.6218 0.000 0.0000 0.000 0.000 0.000   0.00
## [3,] 0.0000 0.598 0.0000 0.000 0.000 0.000   0.00
## [4,] 0.0000 0.000 0.5666 0.000 0.000 0.000   0.00
## [5,] 0.0000 0.000 0.0000 0.488 0.000 0.000   0.00
## [6,] 0.0000 0.000 0.0000 0.000 0.217 0.000   0.00
## [7,] 0.0000 0.000 0.0000 0.000 0.000 0.725   0.00
\end{verbatim}

\begin{Shaded}
\begin{Highlighting}[]
\KeywordTok{library}\NormalTok{(demogR) }\CommentTok{#demogR package allows you to analyze age-structured population models}
\end{Highlighting}
\end{Shaded}

\begin{verbatim}
## Warning: package 'demogR' was built under R version 3.5.2
\end{verbatim}

\begin{Shaded}
\begin{Highlighting}[]
\CommentTok{#help(package="demogR") #help menu for demogR}

\CommentTok{#Eigenvalue analysis}
\NormalTok{analysis<-}\KeywordTok{eigen.analysis}\NormalTok{(aMatrix)}
\CommentTok{#asymptotic growth rate: lambda}
\NormalTok{analysis}\OperatorTok{$}\NormalTok{lambda1}
\end{Highlighting}
\end{Shaded}

\begin{verbatim}
## [1] 1.416025
\end{verbatim}

\begin{Shaded}
\begin{Highlighting}[]
\CommentTok{# 1.416025}
\CommentTok{# Greater than 1 means the population is growing exponentially}

\CommentTok{# Stable age distribution}
\NormalTok{analysis}\OperatorTok{$}\NormalTok{stable.age}
\end{Highlighting}
\end{Shaded}

\begin{verbatim}
## [1] 0.577948350 0.253786641 0.107176347 0.042884913 0.014779283 0.002264864
## [7] 0.001159602
\end{verbatim}

\begin{Shaded}
\begin{Highlighting}[]
\CommentTok{#[1] 0.577948350 0.253786641 0.107176347 0.042884913 0.014779283 0.002264864}
\CommentTok{#[7] 0.001159602}

\CommentTok{# This tells us that most of the population is in the first stage (eggs) and few hatch}
\CommentTok{# and then subsequently make it to adulthood.}
\NormalTok{analysis}\OperatorTok{$}\NormalTok{elasticities}
\end{Highlighting}
\end{Shaded}

\begin{verbatim}
##             A1        A2        A3        A4        A5        A6        A7
## [1,] 0.0000000 0.0000000 0.0000000 0.0000000 0.0000000 0.0000000 0.1428571
## [2,] 0.1428571 0.0000000 0.0000000 0.0000000 0.0000000 0.0000000 0.0000000
## [3,] 0.0000000 0.1428571 0.0000000 0.0000000 0.0000000 0.0000000 0.0000000
## [4,] 0.0000000 0.0000000 0.1428571 0.0000000 0.0000000 0.0000000 0.0000000
## [5,] 0.0000000 0.0000000 0.0000000 0.1428571 0.0000000 0.0000000 0.0000000
## [6,] 0.0000000 0.0000000 0.0000000 0.0000000 0.1428571 0.0000000 0.0000000
## [7,] 0.0000000 0.0000000 0.0000000 0.0000000 0.0000000 0.1428571 0.0000000
## attr(,"class")
## [1] "leslie.matrix"
\end{verbatim}

\begin{Shaded}
\begin{Highlighting}[]
\CommentTok{#             A1        A2        A3        A4        A5        A6        A7}
\CommentTok{# [1,] 0.0000000 0.0000000 0.0000000 0.0000000 0.0000000 0.0000000 0.1428571}
\CommentTok{# [2,] 0.1428571 0.0000000 0.0000000 0.0000000 0.0000000 0.0000000 0.0000000}
\CommentTok{# [3,] 0.0000000 0.1428571 0.0000000 0.0000000 0.0000000 0.0000000 0.0000000}
\CommentTok{# [4,] 0.0000000 0.0000000 0.1428571 0.0000000 0.0000000 0.0000000 0.0000000}
\CommentTok{# [5,] 0.0000000 0.0000000 0.0000000 0.1428571 0.0000000 0.0000000 0.0000000}
\CommentTok{# [6,] 0.0000000 0.0000000 0.0000000 0.0000000 0.1428571 0.0000000 0.0000000}
\CommentTok{# [7,] 0.0000000 0.0000000 0.0000000 0.0000000 0.0000000 0.1428571 0.0000000}
\CommentTok{# attr(,"class")}
\CommentTok{# [1] "leslie.matrix"}

\CommentTok{# The elasticities tells us that any in difference surival among the life stages do not affect}
\CommentTok{#the growth rate of the population. }
\NormalTok{comadre}\OperatorTok{$}\NormalTok{metadata}\OperatorTok{$}\NormalTok{DOI.ISBN[}\DecValTok{845}\NormalTok{]}
\end{Highlighting}
\end{Shaded}

\begin{verbatim}
## [1] "10.1007/BF00349187"
\end{verbatim}

\begin{Shaded}
\begin{Highlighting}[]
\NormalTok{analysis}\OperatorTok{$}\NormalTok{sensitivities}
\end{Highlighting}
\end{Shaded}

\begin{verbatim}
##           [,1]      [,2]      [,3]      [,4]      [,5]      [,6]
## [1,] 0.0000000 0.0000000 0.0000000 0.0000000 0.0000000 0.0000000
## [2,] 0.3253286 0.0000000 0.0000000 0.0000000 0.0000000 0.0000000
## [3,] 0.0000000 0.3382765 0.0000000 0.0000000 0.0000000 0.0000000
## [4,] 0.0000000 0.0000000 0.3570232 0.0000000 0.0000000 0.0000000
## [5,] 0.0000000 0.0000000 0.0000000 0.4145273 0.0000000 0.0000000
## [6,] 0.0000000 0.0000000 0.0000000 0.0000000 0.9322088 0.0000000
## [7,] 0.0000000 0.0000000 0.0000000 0.0000000 0.0000000 0.2790197
##              [,7]
## [1,] 0.0002866303
## [2,] 0.0000000000
## [3,] 0.0000000000
## [4,] 0.0000000000
## [5,] 0.0000000000
## [6,] 0.0000000000
## [7,] 0.0000000000
## attr(,"class")
## [1] "leslie.matrix"
\end{verbatim}

\begin{Shaded}
\begin{Highlighting}[]
\CommentTok{#           [,1]      [,2]      [,3]      [,4]      [,5]      [,6]         [,7]}
\CommentTok{# [1,] 0.0000000 0.0000000 0.0000000 0.0000000 0.0000000 0.0000000 0.0002866303}
\CommentTok{# [2,] 0.3253286 0.0000000 0.0000000 0.0000000 0.0000000 0.0000000 0.0000000000}
\CommentTok{# [3,] 0.0000000 0.3382765 0.0000000 0.0000000 0.0000000 0.0000000 0.0000000000}
\CommentTok{# [4,] 0.0000000 0.0000000 0.3570232 0.0000000 0.0000000 0.0000000 0.0000000000}
\CommentTok{# [5,] 0.0000000 0.0000000 0.0000000 0.4145273 0.0000000 0.0000000 0.0000000000}
\CommentTok{# [6,] 0.0000000 0.0000000 0.0000000 0.0000000 0.9322088 0.0000000 0.0000000000}
\CommentTok{# [7,] 0.0000000 0.0000000 0.0000000 0.0000000 0.0000000 0.2790197 0.0000000000}
\CommentTok{# attr(,"class")}
\CommentTok{# [1] "leslie.matrix"}

\CommentTok{# The sensitivities tell us that stage 5 which is the diapause stage of the Colias alexandra}
\CommentTok{# butterfly, is the most important stage for the growth of the population. }
\CommentTok{#Based on the original study by Hayes, 1981, the author concluded that diapause stage}
\CommentTok{#contributed the most towards population growth because of environmental conditions}
\CommentTok{#such as cold weather which coincides with our analysis.}
\end{Highlighting}
\end{Shaded}


\end{document}
